\documentclass[a4paper]{article}
\usepackage{polski}
\usepackage[utf8]{inputenc}
\usepackage{amsmath}
\usepackage{graphicx}
\usepackage{caption}
\usepackage{subcaption}
\usepackage{geometry}
\usepackage{float}


\title{MGR - Raport 1}
\author{Jakub Postępski}

\begin{document}

\maketitle

\section{Problem}
Należy skompensować siłę grawitacji w układzie ze sprężyną.

\section{Obiekt}
Obiektem jest układ ze sprężyną i amortyzatorem. Na końcu zawieszona jest punktowa masa $m$ której grawitację należy skompensować. Przyjmujemy że dla użytkownika dostępne są pomiary położenia ($x$), prędkości ($\dot{x}$), przyspieszenia ($\ddot{x}$) oraz siły odczytanej przez czujnik.

Model obiektu można opisać układem równań różniczkowych:

\[
\begin{bmatrix}
    \dot{x} \\
    \ddot{x}
\end{bmatrix}
=
\begin{bmatrix}
    0 & 1 \\
    \frac{k}{m} & \frac{b}{m}
\end{bmatrix}
+
\begin{bmatrix}
    0 \\
    -1
\end{bmatrix}
g+
\begin{bmatrix}
    0\\
    1
\end{bmatrix}a
\]

gdzie:
\begin{itemize}
	\item $x$ pozycja zawieszonej masy
	\item $k$ parametr sztywności
	\item $b$ parametr amortyzacji
	\item $m$ masa przedmiotu
	\item $g$ przyspieszenie ziemskie
	\item $a$ przyspieszenie które można dodać do układu
\end{itemize}

Symulację obiektu można wykonać poprzez konwersję do macierzy układu do układu dyskretnego. W pracy przyjęto częstotliwość próbkowania $f = 100 Hz$. Czujnik siły można zasymulować biorąc przyspieszenie układu $\ddot{x}$ i mnożąć przez masę. Do wszystkich wartości dodano szum biały.

\section{Kompensacja masy}
\subsection{Odczyt z czujnika siły}
Siła odczytywana z czujnika siły jest sumą sił działających na masę. Kiedy masa się nie porusza w łatwy sposób można odczytać siłę grawitacji a więc i zawieszoną masę. W momencie ruchu układu z czujnika dostajemy sygnał zmienny. W celu uzyskania interesującej nas składowej stałej (siła grawitacji) możemy zastosować transformatę Fouriera. Jedną z własności tej transformaty jest fakt że uzyskana tak składowa stała jest średnią arytmetyczną sygnału. Podsumowując należy
W celu detekcji masy stosowane są dwie metody, których wynik łączony jest przy pomocy logiki rozmytej. Pierwsza z metod polega na bezpośrednim odczycie



\end{document}