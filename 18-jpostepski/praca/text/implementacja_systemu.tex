% encoding: utf8
% !TEX encoding = utf8
% !TeX spellcheck = pl_PL

\chapter{Implementacja systemu\label{chap:implementacja_systemu}}
W ramach pracy zmodyfikowano istniejący system robotyczny opisany w rozdziale \ref{chap:srodowisko}. Dzięki temu możliwe było spełnienie założeń i wymagań z rozdziału \ref{chap:specyfikacja_systemu}. 


\section{Kompensowanie wpływu grawitacji narzędzia}
Kiedy chwytak robota chwyci narzędzie zmienia się łańcuch kinematyczny. W pracy zajmujemy się przypadkiem w którym po uchwyceniu narzędzie zostaje nieruchome w stosunku do chwytaka. Można przyjąć, że zmieniają się wtedy parametry ostatniego członu związane z masą i inercją.

Głównym problemem w trakcie manipulacji z nieskompensowaną grawitacją jest znaczny uchyb statyczny. Aby wyeliminować ten efekt można do istniejącego algorytmu sterowania impedancyjnego dodać algorytm PID. 

Należy mieć na uwadze, że niektóre człony są takie same w zaprezentowanych prawach sterowania. W impedancyjnym prawie sterwowania nie ma członu całkującego i został on skopiowany z algorytmu PID. W rezultacie nowe prawo sterowania jest postaci:

\begin{equation}
\boldsymbol{\mathcal{F}} = \boldsymbol{K_x}\boldsymbol{e_x} + \boldsymbol{D_x}\dot{\boldsymbol{e_x}} + \int_{0}^{t}  \boldsymbol{I}\boldsymbol{e_x}dt
\end{equation}

gdzie:
\begin{itemize}
    \item $\boldsymbol{\mathcal{F}}$ to wektor sił wynikowych
    \item $\boldsymbol{K_x}$ to diagonalna macierz sprężystości
    \item $\boldsymbol{D_x}$ to diagonalna macierz sztywności
    \item $\boldsymbol{I}$ to diagonalna macierz członu całkującego
    \item $\boldsymbol{e_x}$ to wektor uchybu
\end{itemize}