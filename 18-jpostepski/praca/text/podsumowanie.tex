% encoding: utf8
% !TEX encoding = utf8
% !TeX spellcheck = pl_PL

\chapter{Podsumowanie\label{chap:podsumowanie}}

Zaprezentowana modyfikacja prawa sterowania impedancyjnego w~przestrzeni operacyjnej pozwoliła na szybkie i~proste rozwiązanie problemu kompensacji grawitacji chwytanego narzędzia. W~trakcie manipulacji nawet skomplikowane ruchy z~chwyconymi przedmiotami o~znacznej masie nie stanowią problemu dla robota. Najbardziej wymagające dla nowego prawa sterowania jest podnoszenie przedmiotów, czyli manipulacja przy zmieniających się paraemtrach narzędzia. Potrzeba adaptacji do narzędzia o~nowych parametrach trwa kilka sekund. 


W przyszłości można usprawnić algorytm poprzez dodanie ograniczenia członu całkującego zapożyczonego z~prawa sterowania PID. Dzięki takiej modyfikacji nie będzie miało miejsce praktycznie żadne pogorszenie uginania się robota w~trakcie kolizji. Jako alternatywę można stosować algorytm który działa tylko w~pionowej osi układu. Prawdopodobne jest jednak to, że spowoduje to znaczne pogorszenie jakości osiąganych trajektorii.  Dalsze badania mogłyby skupić się na przyjęciu uproszczonego modelu robota wraz z~narzędziem oraz estymacji jego parametrów z~wykorzystaniem czujnika FTS i~wyliczonych pozycji końcówki.

Zaprezentowana metodyka pracy stanowi dobre podstawy do testowania zupełnie nowych i~bardziej wysublimowanych algorytmów.  Badania w~obszarze metodyki powinny skupić się na stworzeniu bardziej precyzyjnych narzędzia do oceny jakości kolizji robota z~otoczeniem.  