% encoding: utf8
% !TEX encoding = utf8
% !TeX spellcheck = pl_PL

\chapter{Wstęp\label{chap:wstep}}

	Współczesne roboty z~powodzeniem zastępują człowieka przy wielu żmudnych i~niewdzięcznych czynnościach.  W~zakładach produkcyjnych sprawują się znakomicie ale do współpracy robotów z~ludźmi podchodzi się niezwykle ostrożnie. Niechęć do tego typu działań podyktowana jest nie tylko wysokimi kosztami ale także obawami związanymi z~ochroną otoczenia, w~którym robot ma operować.
	
	Zakres zastosowań robotyki można znacznie poszerzyć projektując roboty bezpieczne zarówno dla otaczającego je środowiska jak i~samych robotów. Jednym z~fundamentów takiego bezpieczeństwa może być algorytm sterowania impedancyjnego, który znacznie ogranicza ryzyko zniszczeń. 
	
	Prawo sterowania jest skonstruowane tak by silniki ramienia działały w~sposób symulujący układ ze sprężyną i~amortyzatorem. Robot sterowany w~taki sposób nie dąży do zadanej pozycji za wszelką cenę. W~momencie kolizji robot pozostaje elastyczny i~ugina się. Opisana zaleta może przerodzić się w~wadę. Algorytm korzysta ze zdefiniowanego modelu, który nie uwzględnia wszystkich cech ramienia oraz chwytanych przez robota przedmiotów. Z~punktu widzenia prawa sterowania chwycony przedmiot jest traktowany tak samo jak reszta środowiska. Zamiast skompensować siłę grawitacji tego przedmiotu ramię robota ugina się pod ciężarem. Dalsza manipulacja ramieniem najczęściej nie ma 
	sensu. 

	Celem pracy jest rozwiązanie problemu kompensacji siły grawitacji przedmiotu o~nieznanej masie i~inercji w~opisanym środowisku. 
	Do badań zostanie wykorzystany robot Velma oraz jego symulator skonstruowane przez Zespół Programowania Robotów i~Systemów Rozpoznających. W~niniejszej pracy zostanie zaprezentowana praktyczna realizacja sposobu kompensacji grawitacji chwyconego narzędzia wraz z~testami.
