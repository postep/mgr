% encoding: utf8
% !TEX encoding = utf8
% !TeX spellcheck = pl_PL

\chapter{Wstęp\label{chap:wstep}}

	Współczesne roboty z powodzeniem zastępują człowieka przy wielu żmudnych i niewdzięcznych czynnościach.  W zakładach produkcyjnych sprawują się znakomicie. Mimo tego podchodzi się niezwykle ostrożnie do współpracy robotów z ludzmi. Podyktowane to jest nie tylko wysokimi kosztami ale także obawami związanymi z ochroną otoczenia w którym robot ma operować.
	
	Zakres zastosowań robotyki można znacznie poszerzyć projektując roboty bezpieczne zarówno dla otaczającego je środowiska jak i samych robotów. Jednym z fundamentów takiego bezpieczeństwa może być algorytm sterowania impedancyjnego, który znacznie ogranicza ryzyko zniszczeń. 
	
	Prawo sterowania algorytmu jest skonstruowane tak by silniki ramienia działały w sposób symulujący układ ze sprężyną i amortyzatorem. Robot sterowany w taki sposób nie minimalizuje uchybu pozycji za wszelką cenę. W momencie kontaktu z otoczeniem robot pozostaje elastyczny i ugina się. Opisana zaleta może przerodzić się w wadę. Algorytm korzysta ze zdefiniowanego modelu który nie uwzględnia chwytanych przez robota przedmiotów. Z punktu widzenia algorytmu sterowania chwycony przedmiot jest traktowany tak samo jak reszta środowiska. Zamiast skompensować siłę grawitacji chwyconego przedmiotu ramię robota ugina się pod ciężarem. 

	Celem pracy jest znalezienie metody kompensacji siły grawitacji przedmiotu o nieznanej masie i inercji w opisanym środowisku. W niniejszej pracy zostanie przeprowadzona dyskusja na temat możliwych sposobów radzenia sobie ze wspomnianymi wadami algorytmu sterowania impedancyjnego. Zostanie zaprezentowana praktyczna realizacja jednego z nich wraz z testami.
