% encoding: utf8
% !TEX encoding = utf8
% !TeX spellcheck = pl_PL

\chapter{Podstawy teoretyczne\label{chap:przeglad_literatury}}
Dyskutowane w pracy algorytmy znane są w wielu różnych odmianach. Poniżej zaprezentowano wersje używane w trakcie dalszych badań. 

Do dalszych rozważań możemy zdefiniować uchyb jako:
\begin{equation}
	\boldsymbol{e_x} = \boldsymbol{x_d} - \boldsymbol{x}
\end{equation}
gdzie:
\begin{itemize}
\item $\boldsymbol{x_d}$ to wektor zadanych pozycji uogólnionych
\item $\boldsymbol{x}$ to wektor pozycji uogólnionych
\end{itemize}

\section{Sterowanie impedancyjne}
Prawo sterowania impedancyjnego w przestrzeni operacyjnej sprawia, że chwytak robota zachowuje się jak przytwierdzony do układu ze sprężyną i amortyzatorem. Pojawienie się nieprzewidzianych sił zewnętrznych powoduje, że uchyb pozycji nie jest minimalizowany za wszelką cenę. Przy kontakcie z otoczeniem stawy robota w pewnym stopniu sprężyste i miękkie. W konsekwencji robot ugina się przed otaczającym go środowiskiem. 

\subsection{Przypadek prosty}
W najprostszym przypadku możemy więc opisać takie prawo jako układ: 

	\begin{equation}
	F = kx + d\dot{x} + F_{ext}
	\end{equation}

gdzie:
\begin{itemize}
\item $F$ to siła wynikowa
\item $k$ to parametr sztywności
\item $d$ to parametr tłumienia
\item $F_{ext}$ to nieznana siła zewnętrzna działająca na układ
\end{itemize} 

\subsection{Prawo sterowania}
\label{sec:impedancyjne}

Można sformułować prawo sterowania jako wektor siły uogólnionej:
\begin{equation}
\boldsymbol{\mathcal{F}} = \boldsymbol{K_x}\boldsymbol{e_x} + \boldsymbol{D_x}\dot{\boldsymbol{e_x}}
\end{equation}

gdzie:
\begin{itemize}
\item $\boldsymbol{K_x}$ to diagonalna macierz sprężystości
\item $\boldsymbol{D_x}$ to diagonalna macierz tłumienia
\item $\boldsymbol{\mathcal{F}_{ext}}$ to wektor nieznanych uogólnionych sił zewnętrznych
\end{itemize}

\subsection{Sterowanie w przestrzeni konfiguracyjnej}
W rzeczywistym ramieniu robotycznym zadajemy momenty na poszczególne stawy robota. Opisane w podrozdziale \ref{sec:impedancyjne} prawo sterowania opisuje przestrzeń operacyjną $\boldsymbol{\mathcal{F}}$. Można uzyskać porządane wartości wektora momentów $\boldsymbol{\tau}$ które zadajemy silnikom w stawach stawie wyliczamy z jakobianu $\boldsymbol{J}$:

\begin{equation}
\boldsymbol{\tau} = \boldsymbol{J}^T(\boldsymbol{q})\boldsymbol{\mathcal{F}}
\end{equation}

\section{Sterowanie PID}
PID jest bardzo popularnym algorytmem sterowania automatycznego. Podstawowym celem algorytmu jest minimalizacja uchybu. Uchyb statyczny jest minimalizowany za wszelką cenę nawet jeśli miałoby to spowodować uszkodzenia. Uchyby dynamiczne nie są już tak dobrze kompensowane.

Człon proporcjonalny algorytmu pozwala na wzmocnienie uchybu i w ten sposób odjęcie go od sygnału sterującego. Człon całkujący algorytmu sumuje przeszłe błędy i odejmuje od sterowania ich sumę. Człon różniczkujący wzmacnia sygnał sterujący w gdy wartość błędu zmienia się w celu przyspieszenia regulacji.
\subsection{Przypadek prosty}
W jednowymiarowym przypadku prawo sterowania jest postaci:
\begin{equation}
F = Pe + I\int_{0}^{t}e dt + D\frac{de}{dt}
\end{equation}

gdzie:
\begin{itemize}
	\item $e$ to uchyb
	\item $P$ to parametr członu proporcjonalnego
	\item $I$ to parametr członu całkującego
	\item $D$ to parametr członu różniczkującego
\end{itemize}

\subsection{Przypadek wielowymiarowy}
Rozpatrując wektor siły uogólnionej możemy założyć że prawo sterowania rozpatruje każdą z wartości wektora niezależnie. Prawo sterowania można zapisać w postaci:
\begin{equation}
\boldsymbol{\mathcal{F}} = \boldsymbol{P}\boldsymbol{e_x} +\int_{0}^{t}  \boldsymbol{I}\boldsymbol{e_x}dt + \boldsymbol{D}\dot{\boldsymbol{e_x}}
\end{equation}
gdzie:
\begin{itemize}
	\item $\boldsymbol{P_x}$ to diagonalna macierz proporcjonalności
	\item $\boldsymbol{I_x}$ to diagonalna macierz członu całkowania
	\item $\boldsymbol{D_x}$ to diagonalna macierz członu różniczkującego
\end{itemize}

\section{Estymacja siły uogólnionej w końcówce}
Dla ramienia robotycznego możemy opisać siły występujące w samym ramieniu zgodnie ze wzorem:
\begin{equation}
\boldsymbol{\mathcal{F}_m}(\boldsymbol{x}, \dot{\boldsymbol{x}}, \ddot{\boldsymbol{x}}, \boldsymbol{q}, \dot{\boldsymbol{q}}) = \boldsymbol{\Lambda}(\boldsymbol{q})\boldsymbol{\ddot{x}} + \boldsymbol{\mu}(\boldsymbol{x}, \boldsymbol{\dot{x}}) + \boldsymbol{\gamma}(\boldsymbol{q}) + \boldsymbol{\eta}(\boldsymbol{q}, \boldsymbol{\dot{q}}) + \boldsymbol{\mathcal{F}_{ext}}
\label{eq:ramie}
\end{equation}

gdzie:
\begin{itemize}
	\item $\boldsymbol{\mathcal{F}}$ to wektor sił wynikowych
	\item $\boldsymbol{\Lambda}$ to dodatnio określona macierz inercji w przestrzeni zadań
	\item $\boldsymbol{\mu}$ to macierz sił Coriolisa i sił odśrodkowych	
	\item $\boldsymbol{\gamma}$ to wektor sił grawitacji
	\item $\boldsymbol{\eta}$ to macierz sił tarcia oraz nieuwzglęnionych sił
	\item $\boldsymbol{q}$ to wektor położeń stawów w przestrzeni konfiguracyjnej
	\item $\boldsymbol{x}$ to wektor położeń końcówki w przestrzeni zadań
	\item $ \boldsymbol{\mathcal{F}_{ext}}$ to nieznany wektor sił zewnętrznych działających na układ
\end{itemize} 

Przy wyliczaniu estymowaniu sił działających na końcówkę należy pamiętać, że w końcówce występują siły wygenerowane przez prawo sterowania oraz rzeczywiste siły występujące w układzie. Wzór estymujący rzeczywistą wartość siły uogólnionej w końcówce można zapisać jako:
\begin{equation}
\boldsymbol{\mathcal{\hat{F}}} = \boldsymbol{\mathcal{F}} + \boldsymbol{\mathcal{F}_m}(\boldsymbol{x}, \dot{\boldsymbol{x}}, \ddot{\boldsymbol{x}}, \boldsymbol{q}, \dot{\boldsymbol{q}})
\end{equation}

gdzie $\boldsymbol{\mathcal{F}}$ to wektor sił uogólnionych wyliczony prawem sterowania.

QUESTION: czy rozwijac ta mysl skoro teog nie implementuje?

\section{Ocena jakości algorytmów sterowania}
\subsection{Jakość sterowania}
\label{chap:ape}
Prostym sposobem oceny jakości algorytmu sterowania jest konfrontacja rzeczywistych pozycji ramienia robota z zadanymi. W pracy przyjęto metrykę APE (ang. Absolute Trajectory Error). Metryka jest popularnym wskaźnikiem testowania algorytmów SLAM (ang. Simultaneous Localization and Mapping) ale może być też użyta do porównania trajektorii zadanej przez interpolator i rzeczywistej. 

Dwie trajektorie są opisane w postaci list wektorów sił uogólnionych $\boldsymbol{P}_{1..n}$ oraz $\boldsymbol{Q}_{1..n}$ gdzie $n$ to ilość próbek. Dla każdej chwili czasowej $i$ jest wyliczany błąd postaci:
\begin{equation}
\boldsymbol{E}_i = \boldsymbol{Q}_i^{-1}\boldsymbol{S}\boldsymbol{P}_i
\end{equation}
Macierz $\boldsymbol{S}$ jest optymalnym w sensie metody najmniejszych kwadratów rzutowaniem wektora $\boldsymbol{Q}_i$ na wektor $\boldsymbol{P}_i$ znalezionym za pomocą metody Horna TODO: cyowania. 

Błąd całkowity jest wyliczany jako błąd średniokwadratowy:
\begin{equation}
RMSE(\boldsymbol{E}_{1..n}) = \sqrt{\frac{1}{n}\sum_{i=1}^{n}||\boldsymbol{E}_i||^2}
\end{equation}.

\subsection{Jakość kontaktu z otoczeniem \label{chap:ocenakonaktu}}

Największą zaletą sterowania impendancyjnego jest ugięcie ramienia robota w momencie kolizji ze środowiskiem. Ocena jakości tej cechy może być opisana jako odchylenie pozycji stawów $\boldsymbol{q}_i$ w stosunku do ustalonej pozycji sprzed kolizji $\boldsymbol{q}$ w chwili $i$.


Błąd całkowity jest wyliczany jako błąd średniokwadratowy:
\begin{equation}
RMSE(\boldsymbol{R}_{1..n}) = \sqrt{\frac{1}{n}\sum_{i=1}^{n}||\boldsymbol{q_d}-\boldsymbol{q}_i||^2}
\end{equation}.

\section{Teoria agenta upostaciowionego}
Agentem możemy nazwać jednostkę która jest: zdolna do komunikowania się ze środowiskiem, zdolna do monitorowania swego otoczenia i podejmowania autonomicznych decyzji. Taka definicja gwarantuje spełnienie wielu cech których oczekujemy od nowoczesnych systemów robotycznych. Z definicji agenty są w stanie samodzielnie reagować na zmiany zachodzące w środowisku w sposób inteligentny. Rozszerzeniem teorii agentowej jest teoria agenta upostaciowionego. Agent upostaciowiony cechuje się tym czym zwykły agnet oraz posiada fizyczne ciało.


W teorii agenta upostaciowionego występują pojęcia rzeczywistych efektorów i receptorów. Służą one odpowiednio do oddziaływania na środowisko i do pozyskiwania wiedzy o tym środowisku. W praktyce oznacza to, że rzeczywistym efektorem może być silnik a rzeczywistym receptorem enkoder. 

Agent $a$ może się składać z trzech podsytemów:
\begin{itemize}
	\item Podsystemu sterowania $c$ który zajmuje się podejmowaniem decyzji na podstawie danych z innych podsystemów. Agent może mieć tylko jeden podsystem sterowania.
	\item Wirtualnego efektora $e$ który pośredniczy w komunikacji pomiędzy podsystemem sterowania i rzeczywistym efektorem.
	\item Wirtualnego receptora $r$ który pośredniczy w komunikacji pomiędzy podsystemem sterowania i rzeczywistym receptorem.
\end{itemize}
Podsystemy składają się z komponentów czyli algorytmów. Każdy z podystemów może mieć wiele komponentów i nie wszystkie muszą być uruchomione w konkretnej chwili. Agent posiada też zdefiniowaną maszynę stanów które mogą się zmieniać przy pomocy funkcji przejścia (predykatów). Stan maszyny stanów definiuje tak zwane zachowanie agenta czyli sposób reakcji podsystemu sterowania i uruchomione komponenty. Dodatkowo agent ma możliwość wymiany danych innymi agentami oraz rzeczywistymi efektorami i receptorami poprzez bufory transmisyjne $T$. 