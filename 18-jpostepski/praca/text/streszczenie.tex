% encoding: utf8
% !TEX encoding = utf8
% !TeX spellcheck = pl_PL

% STRONA TYTUŁOWA

\includepdf{pdf/strona_tytulowa.pdf}
\clearpage\mbox{}\thispagestyle{empty}\newpage


% AUTOR


%\par
%\vspace{0.2\baselineskip}
%\hfill\parbox{15em}{{\small\dotfill}\\[-.3ex]
%	\centerline{\footnotesize podpis studenta}}\par
%
%\vspace{1\baselineskip}

\clearpage\mbox{}\newpage


% STRESZCZENIE POLSKIE

\vspace*{\baselineskip}
\begin{center}
	{\large\bfseries Streszczenie}\par\bigskip
\end{center}
\noindent{\bf Tytuł}: {\itshape Chwytanie przedmiotów w robocie sterowanym impedancyjnie}
\\\\
{
	Celem pracy jest implementacja algorytmu kompensującego wpływ siły grawitacji chwytanego przedmiotu o nieznanych parametrach masy i inercji w robocie sterowanym impedancyjnie.
	
	Wykorzystywany do badań robot Velma zbudowany jest z dwóch ramion LWR-4 osadzonych na ruchomym korpusie. Oprogramowanie robota zostało zaprojektowane przy pomocy teorii agenta upostaciowionego i zaimplementowane przy użyciu struktury ramowej FABRIC. Robot korzysta z impedancyjnego prawa sterowania i samodzielnie kompensuje siłę grawitacji ramion. 
}\\\\
\vspace*{0.6\baselineskip}
\noindent{\bf Słowa kluczowe}: {\itshape Velma, FABRIC, ROS, LWR}

\clearpage\mbox{}\newpage


% STRESZCZENIE ANGIELSKIE
\vspace*{\baselineskip}
\begin{center}
	{\large\bfseries Abstract}\par\bigskip
\end{center}
\noindent{\bf Title}: {\itshape Grasping objects in impedance control robot}
\\\\
{ 
	The aim of this thesis is algorithm with gravity compensation 
}\par
\vspace*{1\baselineskip}
\noindent{\bf Keywords}: {\itshape grasping, localization, IRp-6 robot, ROS, DisCODe, OpenCV}

\clearpage\mbox{}\newpage


% OSWIADCZENIE
\includepdf[pagecommand={}]{pdf/oswiadczenie.pdf}
\clearpage\mbox{}\newpage


% PODZIĘKOWANIA


