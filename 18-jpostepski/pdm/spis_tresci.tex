% encoding: utf8
% !TEX encoding = utf8
% !TeX spellcheck = pl_PL


\documentclass[12pt,a4paper]{article}
\usepackage[margin=1.2in]{geometry}
\usepackage{amsmath}
\usepackage{amssymb}
\usepackage[english,polish]{babel}
\usepackage{cite}
\usepackage{graphicx}
\usepackage{hyperref}
\usepackage[utf8]{inputenc}
\usepackage{listings}
\usepackage{polski}
\usepackage{url}
\usepackage{float}
\usepackage[nottoc]{tocbibind}


\graphicspath{ {./img/} }


\lstset{
	basicstyle=\ttfamily,
	columns=fullflexible,
	frame=single,
	breaklines=true,
	postbreak=\mbox{{$\hookrightarrow$}},
}


\begin{document}
	\title{Sterowanie ramieniem robota w obliczu chwytania przedmiotów}
	\author{Jakub Postępski, kierunek AiR, 257947 \\ opiekun pracy: dr inż. Tomasz Winiarski}
	\maketitle

	\tableofcontents
\addtocontents{toc}{\protect\thispagestyle{empty}}
\pagenumbering{gobble}
	\newpage

	\section{Wstęp}
	\section{Robot usługowy Velma}
	\section{Algorytmy kompensacji grawitacji}
	\subsection{Algorytm estymacji parametrów}
	\subsection{Algorytm kompensacji uchybu}
	\section{Realizacja praktyczna algorytmu kompensacji grawitacji}
	\subsection{Wybór algorytmu}
	\subsection{Implementacja algorytmu}
	\section{Testy}
	\subsection{Przypadek testowy}
	\subsection{Uruchomienie bez algorytmu kompensacji grawitacji}
	\subsection{Uruchomienie z załączonym algorytmem kompensacji grawitacji}
	\subsection{Porównanie wyników}
	\section{Podsumowanie}


\end{document}


